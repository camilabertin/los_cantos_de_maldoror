\documentclass[11pt]{article}
\begin{document}

{\huge
Allí teneis a la loca que pasa bailando, mientras rememora vagamente algo. Los niños la persiguen a pedradas como si fuera un mirlo. Enarbola un palo, y hace ademán de correrlos; luego prosigue su camino. Ha perdido un zapato en el trayecto, pero no lo nota. Largas patas de araña recorren su nuca: no son tan sólo sus cabellos. Su rostro ha dejado de parecerse a un rostro humano, y lanza carcajadas como la hiena. Se le escapan jirones de frases, en las que, por más que se las hilvane, muy pocos encontrarían un significado claro. Su vestido, con agujeros en más de un sitio, está animado de violentas sacudidas en tornos de sus piernas huesudas y embarradas. Ella marcha hacia adelante como la hoja del álamo, viéndose arrastrada, ella, su juventud, sus ilusiones y su felicidad pasada que vuelve a ver a través de las brumas de una inteligencia destruida, por el torbellino de las facultades inconscientes. Ha perdido su encanto y su belleza primeros; su andar es grosero y su aliento hiede a aguardiente. Si los hombres fueran felices en esta Tierra, entonces sería la ocasión para asombrarse. La loca no hace ningún reproche, es demasiado altiva para quejarse, y morirá sin haber revelado su secreto a los que se interesan por ella, pero a quienes a prohibido que le dirijan la palabra. Los niños le persiguen a pedradas como si fuera un mirlo. Se le acaba de caer del seno un rollo de papel. Un desconocido lo recoge, se encierra en su casa toda la noche y lee el manuscrito que contiene lo que sigue: "...

}


\end{document}
